\phantomsection
\section*{VII. Conclusion: The Mission and the Mirror}
\addcontentsline{toc}{section}{VII. Conclusion}
\label{sec:vii-conclusion}

Captain Willard's line---``Everyone gets everything he wants. I wanted a mission, and for my
sins they gave me one''---articulates a structure that multiple traditions diagnosed long
before Coppola filmed it. The convergence is striking: frameworks that share no common
metaphysics---biblical justice, Buddhist causality, Western existentialism, critical theory,
depth psychology---independently arrive at the same pattern. Conrad's literary structure
(Section II) provides the journey's template; theological and Buddhist traditions (Section
III) read fulfillment as moral or karmic disclosure; twelve Western philosophers (Section IV)
diagnose the will's metaphysical, existential, and ethical failures; critical theorists
(Section V) historicize those failures within colonial bureaucracy; psychologists (Section VI)
reveal desire as death-denial. Each framework offers a different explanation of \emph{why}
fulfillment punishes---but all agree that it does.

Yet the traditions converge on a single claim: the problem is not that desire is denied but
that it is granted. Punishment comes not from deprivation but from delivery. This is the
film's thesis, enacted through Willard's journey: each fulfilled objective (the
\hyperref[scene:kilgore-beach]{beach secured}, the \hyperref[scene:sampan]{protocol followed},
the \hyperref[scene:assassination]{mission completed}) strips another alibi until only the
will's complicity remains visible. The horror is not what happens but what wanting it reveals about
the will that wanted.

None of these traditions counsel resignation. Each proposes a discipline of wanting: Augustine's
rightly ordered love, the Buddha's Noble Path, Kant's categorical imperative, Beauvoir's
reciprocity, Levinas's responsibility for the face, Arendt's plural action, Frankl's search
for meaning beyond the self. The common thread is reformation: not the annihilation of desire
but its reorientation away from possession and toward responsibility, away from fantasy and
toward lucidity.

The film offers no such reformation. \hyperref[scene:assassination]{Willard completes the
mission} and inherits \hyperref[scene:kurtz-compound]{Kurtz's place}, but nothing changes. The
will remains trapped in the structure it exposed. The traditions converge on the wound; they
diverge on whether it can heal. Augustine promises grace, the Buddha promises cessation, Kant
promises duty, Beauvoir promises reciprocity, Frankl promises meaning. The film promises
nothing. Willard departs in silence, having seen what his wanting was, with no indication that
he can want differently.

Perhaps that is the final teaching: diagnosis is not cure. The essay has demonstrated that
multiple traditions independently named the structure Coppola dramatizes---\emph{delivery, not
deprivation, is the wound}. But naming the wound does not close it. The mission is the mirror.
What the will does with the reflection, the film does not say.
